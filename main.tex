\documentclass[10pt]{article}

% landscape, twocolumn
\usepackage[a4paper, landscape, twocolumn, twoside]{geometry}
% \usepackage[a4paper, top = 0.6in, bottom = 0.3in, left = 0.5in, right = 0.5in, landscape, twocolumn, twoside]{geometry}

\usepackage{calc}
\setlength{\topmargin}{-1in + 15pt}
\setlength{\headheight}{12pt}
\setlength{\headsep}{10pt}
\setlength{\footskip}{0pt}
\setlength{\textheight}{\paperheight - 60pt}

\setlength{\oddsidemargin}{-1in + 30pt}
\setlength{\evensidemargin}{-1in + 30pt}
\setlength{\textwidth}{\paperwidth - 60pt}

\usepackage[xetex, colorlinks]{hyperref}
\usepackage{fontspec, xunicode, xltxtra}
\usepackage{graphicx}
\usepackage{listings}
\usepackage{xcolor}
\usepackage{color}
\usepackage{amsmath}
\usepackage{amssymb}
\usepackage{fancyhdr}


\setmainfont{AR PL UKai CN}


\setlength{\columnsep}{0.4in}

\pagestyle{fancy}
\fancyhead[LE,RO]{\bfseries\thepage}
\fancyhead[LO,RE]{\bfseries\leftmark}
\fancyhead[C]{\bfseries RogerRo}
\fancyfoot{}

% \newfontfamily{\lsttype}{Liberation Mono}

\definecolor{dkgreen}{RGB}{63, 127, 85}
\definecolor{dkpurple}{RGB}{127, 0, 85}
\definecolor{dkgrey}{RGB}{127, 127, 127}
\definecolor{dkred}{RGB}{154, 0, 0}
\lstset {
  basicstyle = \small\ttfamily,
  language = C++,
  aboveskip = 0pt,
  numbers = left,
  numberstyle = \footnotesize\ttfamily\color{dkgrey},
  numbersep = 5pt,
  tabsize = 2,
  breaklines = true,
  breakindent = 1.1em,
  keywordstyle = \color{dkpurple},
  commentstyle = \color{dkgreen},
  backgroundcolor = \color{white},
  stringstyle = \color{dkred},
  deletekeywords = {in},
  showspaces = false,
  basewidth = {0.5em, 0.4em},
  frame = trbl,
  rulecolor = \color{dkgrey},
  showstringspaces = false,
  escapeinside = {<TeX>}{</TeX>}
}

\begin{document}
% \title{\LARGE Templates For ACM}
% \date{}
% \author{\textbf{RogerRo}}
% \maketitle
\tableofcontents
\newpage
%==================================================
\section{母版/基础/类/配置/黑科技}
\subsection{一般母版}
\lstinputlisting{/Template/template.cpp}
\subsection{高精度类}
\lstinputlisting{/Template/template_highprecision.cpp}
\subsection{离散化}
\lstinputlisting{/Template/discretization.cpp}
\subsection{Linux对拍}
\lstinputlisting{/Template/match.sh}
\subsection{vimrc}
\lstinputlisting{/Template/vimrc}
%==================================================
\section{数学}
\subsection{筛素数-欧拉筛法}
\mathcal{O}(N)
\lstinputlisting{/Template/euler_sieve.cpp}

\subsection{高阶代数方程求根-求导}
\mathcal{O}(N^3*S),S取决于精度
\lstinputlisting{/Template/equation.cpp}
%==================================================
\section{几何}
\subsection{最小圆覆盖-随机增量}
\mathcal{O}(N)
\lstinputlisting{/Template/min_cover_circle.cpp}
%==================================================
\section{博弈}
%==================================================
\section{DP}
%==================================================
\section{串}
\subsection{多模匹配-AC自动机}
求n个模式串中有多少个出现过,模式串相同算作多个,\mathcal{O}(\sum P_i+T)
\lstinputlisting{/Template/aho_corasick.cpp}
%==================================================
\section{图/树}
\subsection{单源最短路-Dijkstra}
不加堆,\mathcal{O}(V^2+E)
\lstinputlisting{/Template/dijkstra.cpp}
加堆,\mathcal{O}(ElogE+V)
\lstinputlisting{/Template/dijkstra2.cpp}

\subsection{最短路-Floyd}
\mathcal{O}(V^3+E)
\lstinputlisting{/Template/floyd.cpp}

\subsection{单源最短路-SPFA}
不加优化,\mathcal{O}(VE+V^2)=\mathcal{O}(kE)
\lstinputlisting{/Template/spfa.cpp}
SLF+LLL优化,\mathcal{O}(VE+V^2)=\mathcal{O}(kE)
\lstinputlisting{/Template/spfa2.cpp}

\subsection{二分图最大匹配-匈牙利}
\mathcal{O}(VE)
\lstinputlisting{/Template/hungary.cpp}

\subsection{有向图极大强连通分量-Tarjan强连通}
\mathcal{O}(V+E)
\lstinputlisting{/Template/scc_tarjan.cpp}

\subsection{最大流-iSAP}
简版(无BFS,递归,gap,cur),\mathcal{O}(V^2*E)
\lstinputlisting{/Template/isap.cpp}
完全版(有BFS,非递归,gap,cur),\mathcal{O}(V^2*E)
\lstinputlisting{/Template/isap2.cpp}

\subsection{最小生成树-Prim}
不加堆,\mathcal{O}(V+E)
\lstinputlisting{/Template/prim.cpp}
加堆,\mathcal{O}(V+E)
\lstinputlisting{/Template/prim2.cpp}

\subsection{最小生成树-Kruskal}
\mathcal{O}(ElogE+αE)
\lstinputlisting{/Template/kruskal.cpp}

\subsection{树的直径-BFS}
\mathcal{O}(N)
\lstinputlisting{/Template/tree_diameter.cpp}

\subsection{LCA-TarjanLCA}
\mathcal{O}(αN+Q)
\lstinputlisting{/Template/lca_tarjan.cpp}
%==================================================
\section{数据结构}
\subsection{并查集}
\lstinputlisting{/Template/disjoint_set.cpp}

\subsection{区间和_单点修改区间查询-树状数组}
\mathcal{O}(NlogN+QlogN)
\lstinputlisting{/Template/bit.cpp}

\subsection{区间和_区间修改单点查询-树状数组}
\mathcal{O}(NlogN+QlogN)
\lstinputlisting{/Template/bit2.cpp}

\subsection{区间和-线段树}
\mathcal{O}(NlogN+QlogN)
\lstinputlisting{/Template/segment_tree.cpp}

\subsection{区间第k大_无修改-主席树}
\mathcal{O}(NlogN+QlogN)
\lstinputlisting{/Template/persistent_tree.cpp}

\subsection{RMQ-ST}
\mathcal{O}(NlogN)~\mathcal{O}(1)
\lstinputlisting{/Template/st.cpp}
%==================================================
\section{其它}
%==================================================
\end{document}
