\documentclass[10pt]{article}

\usepackage[a4paper, landscape, twocolumn, twoside]{geometry}

\usepackage{calc}
\setlength{\topmargin}{-1in + 15pt}
\setlength{\headheight}{12pt}
\setlength{\headsep}{10pt}
\setlength{\footskip}{0pt}
\setlength{\textheight}{\paperheight - 60pt}

\setlength{\oddsidemargin}{-1in + 30pt}
\setlength{\evensidemargin}{-1in + 30pt}
\setlength{\textwidth}{\paperwidth - 60pt}

\setlength{\columnsep}{0.4in}
\usepackage{xeCJK}
\setsansfont{Source Code Pro}

\setCJKmainfont{Noto Sans CJK JP}
\setCJKsansfont{Noto Sans CJK JP}
\setCJKmonofont{Noto Sans CJK JP}

\usepackage{fancyhdr}
\pagestyle{fancy}
\fancyhead[RE,RO]{\bfseries\thepage}
\fancyhead[LE,LO]{\bfseries\leftmark}
\fancyhead[C]{\bfseries RogerRo}
\fancyfoot{}


\usepackage{listings}
\lstset{
	language=C++,
	breaklines=true,
	tabsize=2,
	basicstyle=\sf\footnotesize,
	numberstyle=\sf\footnotesize,
	commentstyle=\sf\footnotesize,
	numbers=left,
	frame=trbl,
	showstringspaces = false,
	escapeinside=``,
	extendedchars=false
}

\usepackage{geometry}
%\geometry{left=2cm,right=1cm,top=1.2cm,bottom=1.5cm,headsep=0.1cm,footnotesep=0.1cm}
\usepackage{courier}

\usepackage[
	CJKbookmarks=true,
	colorlinks,
	linkcolor=black,
	anchorcolor=black,
	citecolor=black
]{hyperref}
\AtBeginDvi{\special{pdf:tounicode UTF8-UCS2}}

\usepackage{sectsty}
\sectionfont{\large}
\subsectionfont{\normalsize}
\subsubsectionfont{\small}

\usepackage{amsmath}
\usepackage{booktabs}

\usepackage{indentfirst}
\setlength{\parindent}{0em}

\begin{document}
% \title{\LARGE Templates For ACM}
% \date{}
% \author{\textbf{RogerRo}}
% \maketitle
\tableofcontents
\newpage
%==================================================
\section{基础/配置/黑科技}
\subsection{一般母版}
\lstinputlisting{Template/template.cpp}

\subsection{黑科技}
\lstinputlisting{Template/black_magic.cpp}

\subsection{位运算}
\lstinputlisting{Template/bithacks.cpp}

\subsection{离散化}
\lstinputlisting{Template/discretization.cpp}

\subsection{Linux对拍}
\lstinputlisting{Template/match.sh}

\subsection{vimrc}
\lstinputlisting{Template/vimrc}
%==================================================
\section{数学}
\subsection{高精度类}
\lstinputlisting{Template/template_highprecision.cpp}

\subsection{筛素数-欧拉筛法}
$O(N)$
\lstinputlisting{Template/euler_sieve.cpp}

\subsection{高阶代数方程求根-求导}
$O(N^3*S)$,S取决于精度
\lstinputlisting{Template/equation.cpp}
%==================================================
\section{几何}
\subsection{平面几何类包}
\lstinputlisting{Template/template_geometry.cpp}
%==================================================
\section{博弈}
\subsection{Nim博弈}
问题:$n$堆石子,每次取一堆中$x$个($x>0$),取完则胜。
奇异态(后手胜):$a_1^a_2^...^a_n=0$

\subsection{Bash博弈}
问题:$n$个石子,每次$x$个($0<x\leq m$),取完则胜。
奇异态(后手胜):$n\equiv 0(mod(m+1))$

\subsection{Wythoff博弈}
问题:2堆石子分别$x, y$个($x>y$),每次取一堆中$x$个($x>0$),或两堆中分别$x$个($x>0$),取完则胜。
奇异态(后手胜):$\left \lfloor \frac{\sqrt{5}+1}{2} (x-y) \right \rfloor =y$

\subsection{Fibonacci博弈}
问题:$n$个石子,先手第一次取$x$个($0<x<n$),之后每次取$x$个($0<x\leq 上一次取数的两倍$),取完则胜。
奇异态(\textbf{先}手胜):$n$不是斐波那契数
%==================================================
\section{DP}
%==================================================
\section{串}
\subsection{最长回文子串-Manacher}
$O(N)$
\lstinputlisting{Template/manacher.cpp}
\subsection{多模匹配-AC自动机}
求n个模式串中有多少个出现过,模式串相同算作多个,$O(\sum P_i+T)$
\lstinputlisting{Template/aho_corasick.cpp}
%==================================================
\section{图/树}
\subsection{单源最短路-Dijkstra}
不加堆,$O(V^2+E)$
\lstinputlisting{Template/dijkstra.cpp}
加堆,$O(ElogE+V)$
\lstinputlisting{Template/dijkstra2.cpp}

\subsection{最短路-Floyd}
$O(V^3+E)$
\lstinputlisting{Template/floyd.cpp}

\subsection{单源最短路-SPFA}
不加优化,$O(VE+V^2)=O(kE)$
\lstinputlisting{Template/spfa.cpp}
SLF+LLL优化,$O(VE+V^2)=O(kE)$
\lstinputlisting{Template/spfa2.cpp}

\subsection{二分图最大匹配-匈牙利}
$O(VE)$
\lstinputlisting{Template/hungary.cpp}

\subsection{有向图极大强连通分量-Tarjan强连通}
$O(V+E)$
\lstinputlisting{Template/scc_tarjan.cpp}

\subsection{最大流-iSAP}
简版(无BFS,递归,gap,cur),$O(V^2*E)$
\lstinputlisting{Template/isap.cpp}
完全版(有BFS,非递归,gap,cur),$O(V^2*E)$
\lstinputlisting{Template/isap2.cpp}

\subsection{最小生成树-Prim}
不加堆,$O(V+E)$
\lstinputlisting{Template/prim.cpp}
加堆,$O(V+E)$
\lstinputlisting{Template/prim2.cpp}

\subsection{最小生成树-Kruskal}
$O(ElogE+αE)$
\lstinputlisting{Template/kruskal.cpp}

\subsection{树的直径-BFS}
$O(N)$
\lstinputlisting{Template/tree_diameter.cpp}

\subsection{LCA-TarjanLCA}
$O(αN+Q)$
\lstinputlisting{Template/lca_tarjan.cpp}
%==================================================
\section{数据结构}
\subsection{并查集}
\lstinputlisting{Template/disjoint_set.cpp}

\subsection{区间和\_单点修改区间查询-树状数组}
$O(NlogN+QlogN)$
\lstinputlisting{Template/bit.cpp}

\subsection{区间和\_区间修改单点查询-树状数组}
$O(NlogN+QlogN)$
\lstinputlisting{Template/bit2.cpp}

\subsection{区间和-线段树}
$O(NlogN+QlogN)$
\lstinputlisting{Template/segment_tree.cpp}

\subsection{区间第k大\_无修改-主席树}
$O(NlogN+QlogN)$
\lstinputlisting{Template/persistent_tree.cpp}

\subsection{RMQ-ST}
$O(NlogN)~O(1)$
\lstinputlisting{Template/st.cpp}
%==================================================
\section{其它}
%==================================================
\end{document}
