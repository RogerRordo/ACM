\documentclass[10pt]{article}

\usepackage[a4paper, landscape, twocolumn, twoside]{geometry}

\usepackage{calc}
\setlength{\topmargin}{-1in + 15pt}
\setlength{\headheight}{12pt}
\setlength{\headsep}{10pt}
\setlength{\footskip}{0pt}
\setlength{\textheight}{\paperheight - 60pt}

\setlength{\oddsidemargin}{-1in + 30pt}
\setlength{\evensidemargin}{-1in + 30pt}
\setlength{\textwidth}{\paperwidth - 60pt}

\setlength{\columnsep}{0.4in}
\usepackage{xeCJK}
\setsansfont{Source Code Pro}
\setCJKmainfont{Noto Sans CJK JP}
\setCJKsansfont{Noto Sans CJK JP}
\setCJKmonofont{Noto Sans CJK JP}

\usepackage{fancyhdr}
\pagestyle{fancy}
\fancyhead[RE,RO]{\bfseries\thepage}
\fancyhead[LE,LO]{\bfseries\leftmark}
\fancyhead[C]{\bfseries RogerRo}
\fancyfoot{}

\usepackage{listings}
\lstset{
	language=C++,
	breaklines=true,
	tabsize=2,
	basicstyle=\sf\footnotesize,
	numberstyle=\sf\footnotesize,
	commentstyle=\sf\footnotesize,
	numbers=left,
	frame=trbl,
	showstringspaces = false,
	escapeinside=``,
	extendedchars=false
}

\usepackage{geometry}
%\geometry{left=2cm,right=1cm,top=1.2cm,bottom=1.5cm,headsep=0.1cm,footnotesep=0.1cm}
\usepackage{courier}

\usepackage[
	CJKbookmarks=true,
	colorlinks,
	linkcolor=black,
	anchorcolor=black,
	citecolor=black
]{hyperref}
\AtBeginDvi{\special{pdf:tounicode UTF8-UCS2}}

\usepackage{sectsty}
\sectionfont{\large}
\subsectionfont{\normalsize}
\subsubsectionfont{\small}

\usepackage{amsmath}
\usepackage{amssymb}
\usepackage{booktabs}

\usepackage{indentfirst}
\setlength{\parindent}{0em}

\begin{document}
% \title{\LARGE Templates For ACM}
% \date{}
% \author{\textbf{RogerRo}}
% \maketitle
\tableofcontents
\newpage
%==================================================
\section{基础/配置/黑科技}
\subsection{一般母版}
\lstinputlisting{Template/template.cpp}

\subsection{黑科技}
\lstinputlisting{Template/black_magic.cpp}

\subsection{位运算}
\lstinputlisting{Template/bithacks.cpp}

\subsection{离散化}
\lstinputlisting{Template/discretization.cpp}

\subsection{Linux对拍}
\lstinputlisting{Template/match.sh}

\subsection{vimrc}
\lstinputlisting{Template/vimrc}

\subsection{gdb}
\lstinputlisting{Template/gdb.cpp}
%==================================================
\section{数学}
\subsection{分数类}
\lstinputlisting{Template/template_fraction.cpp}

\subsection{高精度类}
\lstinputlisting{Template/template_highprecision.cpp}

\subsection{矩阵类}
\lstinputlisting{Template/template_matrix.cpp}

\subsection{筛素数-欧拉筛法}
$O(N)$
\lstinputlisting{Template/euler_sieve.cpp}

\subsection{线性方程组-高斯消元}
$O(N^3)$
\lstinputlisting{Template/gauss_elimination.cpp}

\subsection{高阶代数方程求根-求导}
$O(N^3*S)$,S取决于精度
\lstinputlisting{Template/equation.cpp}
%==================================================
\section{几何}
\subsection{平面几何类包}
下面提到皮克公式:$S=I+\frac{B}{2}-1$描述顶点都在格点的多边形面积,$I, B$分别为多边形内、边上格点
\lstinputlisting{Template/template_geometry.cpp}
%==================================================
\section{DP}
%==================================================
\section{串}
\subsection{最长回文子串-Manacher}
$O(N)$
\lstinputlisting{Template/manacher.cpp}
\subsection{多模匹配-AC自动机}
求n个模式串中有多少个出现过,模式串相同算作多个,$O(\sum P_i+T)$
\lstinputlisting{Template/aho_corasick.cpp}
%==================================================
\section{图/树}
\subsection{单源最短路-Dijkstra}
不加堆,$O(V^2+E)$
\lstinputlisting{Template/dijkstra.cpp}
加堆,$O(ElogE+V)$
\lstinputlisting{Template/dijkstra2.cpp}

\subsection{最短路-Floyd}
$O(V^3+E)$
\lstinputlisting{Template/floyd.cpp}

\subsection{单源最短路-SPFA}
不加优化,$O(VE+V^2)=O(kE)$
\lstinputlisting{Template/spfa.cpp}
SLF+LLL优化,$O(VE+V^2)=O(kE)$
\lstinputlisting{Template/spfa2.cpp}

\subsection{二分图最大匹配-匈牙利}
$O(VE)$
\lstinputlisting{Template/hungary.cpp}

\subsection{有向图极大强连通分量-Tarjan强连通}
$O(V+E)$
\lstinputlisting{Template/scc_tarjan.cpp}

\subsection{最大流-iSAP}
简版(无BFS,递归,gap,cur),$O(V^2*E)$
\lstinputlisting{Template/isap.cpp}
完全版(有BFS,非递归,gap,cur),$O(V^2*E)$
\lstinputlisting{Template/isap2.cpp}

\subsection{最小生成树-Prim}
不加堆,$O(V+E)$
\lstinputlisting{Template/prim.cpp}
加堆,$O(V+E)$
\lstinputlisting{Template/prim2.cpp}

\subsection{最小生成树-Kruskal}
$O(ElogE+αE)$
\lstinputlisting{Template/kruskal.cpp}

\subsection{树的直径-BFS}
$O(N)$
\lstinputlisting{Template/tree_diameter.cpp}

\subsection{LCA-TarjanLCA}
$O(αN+Q)$
\lstinputlisting{Template/lca_tarjan.cpp}
%==================================================
\section{数据结构}
\subsection{并查集}
\lstinputlisting{Template/disjoint_set.cpp}

\subsection{区间和\_单点修改区间查询-树状数组}
$O(NlogN+QlogN)$
\lstinputlisting{Template/bit.cpp}

\subsection{区间和\_区间修改单点查询-树状数组}
$O(NlogN+QlogN)$
\lstinputlisting{Template/bit2.cpp}

\subsection{区间和-线段树}
$O(NlogN+QlogN)$
\lstinputlisting{Template/segment_tree.cpp}

\subsection{平衡树-Treap}
\lstinputlisting{Template/treap.cpp}

\subsection{区间第k大\_无修改-主席树}
$O(NlogN+QlogN)$
\lstinputlisting{Template/persistent_tree.cpp}

\subsection{RMQ-ST}
$O(NlogN)~O(1)$
\lstinputlisting{Template/st.cpp}
%==================================================
\section{其它}
\subsection{n皇后问题-构造}
(输出任一方案),$O(n)$
\lstinputlisting{Template/n_queen.cpp}
%==================================================
\section{纯公式/定理}
\subsection{数学公式}
\subsubsection{三角}
\paragraph{$\blacksquare$ 复分析欧拉公式}
\noindent \\
$$e^{ix}=\cos x+i\sin x$$(可简单导出棣莫弗定理)
\paragraph{$\blacksquare$ 和差公式}
\noindent \\
$\sin(\alpha \pm \beta )=\sin \alpha \cos \beta \pm \cos \alpha \sin \beta$~~~~~~~~
$\cos(\alpha \pm \beta )=\cos \alpha \cos \beta \mp \sin \alpha \sin \beta$\\
$\tan(\alpha \pm \beta )={\frac{\tan \alpha \pm \tan \beta }{1\mp \tan \alpha \tan \beta }}$\\
\paragraph{$\blacksquare$ 和差化积}
\noindent \\
$\sin \alpha +\sin \beta =2\sin {\frac  {\alpha +\beta }{2}}\cos {\frac  {\alpha -\beta }{2}}$~~~~~~~~
$\cos \alpha +\cos \beta =2\cos {\frac  {\alpha +\beta }{2}}\cos {\frac  {\alpha -\beta }{2}}$\\
$\cos \alpha -\cos \beta =-2\sin {\alpha +\beta  \over 2}\sin {\alpha -\beta  \over 2}$~~~~~~~~
$\sin \alpha -\sin \beta =2\cos {\alpha +\beta  \over 2}\sin {\alpha -\beta  \over 2}$\\
\paragraph{$\blacksquare$ 积化和差}
\noindent \\
$\sin \alpha \sin \beta =\frac{\cos(\alpha -\beta )-\cos(\alpha +\beta )}{2}$~~~~~~~~
$\cos \alpha \cos \beta =\frac{\cos(\alpha -\beta )+\cos(\alpha +\beta )}{2}$\\
$\sin \alpha \cos \beta =\frac{\cos(\alpha +\beta )+\cos(\alpha -\beta )}{2}$~~~~~~~~
$\cos \alpha \sin \beta =\frac{\cos(\alpha +\beta )-\cos(\alpha -\beta )}{2}$\\
\paragraph{$\blacksquare$ 二、三、$n$倍角(切比雪夫)}
\noindent \\
$\sin 2\theta =2\sin \theta \cos \theta =\frac{2\tan \theta}{1+\tan ^2 \theta}$\\
$\cos 2\theta =\cos ^2 \theta -\sin ^2 \theta =2\cos ^2 \theta -1=1-2\sin ^2 \theta=\frac{1-\tan ^2 \theta}{1+\tan ^2 \theta}$\\
$\tan 2\theta =\frac{2\tan \theta}{1-\tan ^2 \theta}=\frac{1}{1-\tan \theta}-\frac{1}{1+\tan \theta}$\\
$\sin 3\theta =3\sin \theta-4\sin ^3 \theta$~~~~~~~~
$\cos 3\theta=4\cos ^3 \theta-3\cos \theta$~~~~~~~~
$\tan 3\theta=\frac{3\tan \theta -\tan ^3 \theta}{1-3\tan ^2 \theta}$\\
$\begin{aligned} \sin n\theta &=\sum _{{k=0}}^{n}{\binom  {n}{k}}\cos ^{k}\theta \,\sin ^{{n-k}}\theta \,\sin \left[{\frac  {1}{2}}(n-k)\pi \right] \\&=\sin \theta \sum _{{k=0}}^{{\lfloor {\frac  {n-1}{2}}\rfloor }}(-1)^{k}{\binom  {n-1-k}{k}}(2\cos \theta )^{n-1-2k} \end{aligned}$\\
$\begin{aligned} \cos n\theta &=\sum _{{k=0}}^{n}{\binom  {n}{k}}\cos ^{k}\theta \,\sin ^{{n-k}}\theta \,\cos \left[{\frac  {1}{2}}(n-k)\pi \right] \\&={\frac  {1}{2}}\sum _{{k=0}}^{{\lfloor {\frac  {n}{2}}\rfloor }}(-1)^{k}{\frac  {n}{n-k}}{\binom  {n-k}{k}}(2\cos \theta )^{n-2k} \end{aligned}$\\
\paragraph{$\blacksquare$ 二、三次降幂}
\noindent \\
$\sin ^{2}\theta ={\frac  {1-\cos 2\theta }{2}}$~~~~
$\cos ^{2}\theta ={\frac  {1+\cos 2\theta }{2}}$~~~~
$\sin ^{3}\theta ={\frac  {3\sin \theta -\sin 3\theta }{4}}$~~~~
$\cos ^{3}\theta ={\frac  {3\cos \theta +\cos 3\theta }{4}}$\\
\paragraph{$\blacksquare$ 万能公式}
\noindent \\
$t=\tan \frac{\theta}{2}$~~~$\Rightarrow$~~~
$\sin \theta = \frac{2t}{1+t^2}$~~~~~~
$\cos \theta = \frac{1-t^2}{1+t^2}$~~~~~~~
$\sin \theta = \frac{2t}{1-t^2}$~~~~~~
$\mathrm{d}x=\frac{2}{1+t^2}\mathrm{d}t$\\
\paragraph{$\blacksquare$ 连乘}
\noindent \\
${\displaystyle \prod _{{k=0}}^{{n-1}}\cos 2^{k}\theta ={\frac  {\sin 2^{n}\theta }{2^{n}\sin \theta }}}$~~~
${\displaystyle \prod _{{k=0}}^{{n-1}}\sin \left(x+{\frac  {k\pi }{n}}\right)={\frac  {\sin nx}{2^{{n-1}}}}}$\\
${\displaystyle \prod _{{k=1}}^{{n-1}}\sin \left({\frac  {k\pi }{n}}\right)={\frac  {n}{2^{{n-1}}}}}$~~~
${\displaystyle \prod _{{k=1}}^{{n-1}}\sin \left({\frac  {k\pi }{2n}}\right)={\frac  {{\sqrt  {n}}}{2^{{n-1}}}}}$~~~
${\displaystyle \prod _{{k=1}}^{{n}}\sin \left({\frac  {k\pi }{2n+1}}\right)={\frac  {{\sqrt  {2n+1}}}{2^{n}}}}$\\
${\displaystyle \prod _{{k=1}}^{{n-1}}\cos \left({\frac  {k\pi }{n}}\right)={\frac  {\sin {\frac  {n\pi }{2}}}{2^{{n-1}}}}}$~~~
${\displaystyle \prod _{{k=1}}^{{n-1}}\cos \left({\frac  {k\pi }{2n}}\right)={\frac  {{\sqrt  {n}}}{2^{{n-1}}}}}$~~~
${\displaystyle \prod _{{k=1}}^{n}\cos \left({\frac  {k\pi }{2n+1}}\right)={\frac  {1}{2^{n}}}}$\\
${\displaystyle \prod _{{k=1}}^{{n-1}}\tan \left({\frac  {k\pi }{n}}\right)={\frac  {n}{\sin {\frac  {n\pi }{2}}}}}$~~~
${\displaystyle \prod _{{k=1}}^{{n-1}}\tan \left({\frac  {k\pi }{2n}}\right)=1}$~~~
${\displaystyle \prod _{{k=1}}^{n}\tan {\frac  {k\pi }{2n+1}}={\sqrt  {2n+1}}}$\\
\paragraph{$\blacksquare$ 其它}
\noindent \\
$x+y+z=n\pi \Rightarrow \tan x+\tan y+\tan z=\tan x \tan y \tan z$\\
$x+y+z=n\pi +\frac{\pi}{2} \Rightarrow \cot x+\cot y+\cot z=\cot x \cot y \cot z$\\
$x+y+z=\pi \Rightarrow \sin 2x+\sin 2y+\sin 2z=4\sin x\sin y\sin z$\\
$\sin(x+y)\sin(x-y)=\sin ^{2}{x}-\sin ^{2}{y}=\cos ^{2}{y}-\cos ^{2}{x}$\\
$\cos(x+y)\cos(x-y)=\cos ^{2}{x}-\sin ^{2}{y}=\cos ^{2}{y}-\sin ^{2}{x}$

\subsubsection{重要数与数列}
\paragraph{$\blacksquare$ 幂级数}
\noindent \\
${\displaystyle \sum_{i=1}^n i=\frac{1}{2}n(n+1)}$~~~~
${\displaystyle \sum_{i=1}^n i^2=\frac{1}{3}n(n+\frac{1}{2})(n+1)}$~~~~
${\displaystyle \sum_{i=1}^n i^3=(\sum_{i=1}^n i)^2=\frac{1}{4}n^2 (n+1)^2}$\\
$\begin{aligned} \sum_{i=1}^n i^m &=  {1 \over m+1}\sum_{k=0}^m {m+1 \choose k} b_k (n+1)^{m+1-k} \\ &= {1 \over m+1} \bigg[ (n+1)^{m+1} - 1 - \sum_{i=1}^n \big((i+1)^{m+1} - i^{m+1} - (m+1)i^m\big) \bigg]\end{aligned}$
\paragraph{$\blacksquare$ 几何级数}
\noindent \\
${\displaystyle \sum_{i=0}^n i c^i = { nc^{n+2} - (n+1)c^{n+1} + c \over (c-1)^2}, \quad c \neq 1}$~~~~~~
${\displaystyle \quad \sum_{i=0}^\infty i c^i = {c \over (1 - c)^2}, \quad \vert c \vert < 1}$
\paragraph{$\blacksquare$ 调和级数}
\noindent \\
$H_n$表调和级数,$${\displaystyle H_n=\sum _{k=1}^{n} \frac{1}{k}}$$\\
${\displaystyle \sum_{i=1}^n iH_i = {n(n+1) \over 2}H_n - {n(n-1) \over 4}}$~~~~~~
${\displaystyle \sum_{i=1}^n H_i = (n+1)H_n - n, \quad}$\\
${\displaystyle \sum_{i=1}^n {i \choose m } H_i = { n+1 \choose m+1} \left(H_{n+1} - {1 \over m+1}\right)}$\\
\paragraph{$\blacksquare$ 组合数}
\noindent \\
\begin{tabular}{|c|c|c|c|c|c|c|c|c|c|c|c|c|}
\hline C(i,j)&0&1&2&3&4&5&6&7&8&9&10&11\\
\hline 0&1&&&&&&&&&&& \\
\hline 1&1&1&&&&&&&&&&\\
\hline 2&1&2&1&&&&&&&&&\\
\hline 3&1&3&3&1&&&&&&&&\\
\hline 4&1&4&6&4&1&&&&&&&\\
\hline 5&1&5&10&10&5&1&&&&&&\\
\hline 6&1&6&15&20&15&6&1&&&&&\\
\hline 7&1&7&21&35&35&21&7&1&&&&\\
\hline 8&1&8&28&56&70&56&28&8&1&&&\\
\hline 9&1&9&36&84&126&126&84&36&9&1&&\\
\hline 10&1&10&45&120&210&252&210&120&45&10&1&\\
\hline 11&1&11&55&165&330&462&462&330&165&55&11&1\\
\hline
\end{tabular}\\
${n \choose k} = {n \choose n-k} = {n\over k}{n-1 \choose k-1} = {n -1 \choose k} + {n-1 \choose k-1}$~~~~~~
${n \choose m}{m \choose k} = { n \choose k} { n-k \choose m-k}$\\
${\displaystyle \sum_{k=0}^n {r+k \choose k} = { r+ n+1 \choose n}}$~~~~~~~~~~
${\displaystyle \sum_{k=0}^n {k \choose m} = { n+1 \choose m+1}}$\\
${\displaystyle \sum_{k=0}^n {r \choose k}{s \choose n -k} = {r+s \choose n}}$\\
\paragraph{$\blacksquare$ 第一类斯特林数}
\noindent \\
$\left[\begin{matrix} n \\ k \end{matrix}\right]$表第一类斯特林数,表$n$元素分作$k$个环排列的方法数,$$\left[\begin{matrix} n \\ 0 \end{matrix}\right]=0, \left[\begin{matrix} 1 \\ 1 \end{matrix}\right]=1, \left[\begin{matrix} n \\ k \end{matrix}\right]=\left[\begin{matrix} n-1 \\ k-1 \end{matrix}\right]+(n-1)\left[\begin{matrix} n-1 \\ k \end{matrix}\right]$$\\
\begin{tabular}{|c|c|c|c|c|c|c|c|c|}
\hline s(i,j)&1&2&3&4&5&6&7&8\\
\hline 1&1&&&&&&& \\
\hline 2&1&1&&&&&&\\
\hline 3&2&3&1&&&&&\\
\hline 4&6&11&6&1&&&&\\
\hline 5&24&50&35&10&1&&&\\
\hline 6&120&274&225&85&15&1&&\\
\hline 7&720&1764&1624&735&175&21&1&\\
\hline 8&5040&13068&13132&6769&1960&322&28&1\\
\hline
\end{tabular}\\
$\left[\begin{matrix} n \\ 1 \end{matrix}\right]=(n-1)!$~~~~
$\left[\begin{matrix} n \\ 2 \end{matrix}\right]=(n-1)!H_{n-1}$~~~~
$\left[\begin{matrix} n \\ n-1 \end{matrix}\right]=\binom{n}{2}$~~~~
${\displaystyle \sum_{k=0}^n \left[\begin{matrix} n \\ k \end{matrix}\right] = n!}$\\
${\displaystyle \left[\begin{matrix} n+1 \\ m+1 \end{matrix}\right] = \sum_k \left[\begin{matrix} n \\ k \end{matrix}\right] {k \choose m} = n! \sum_{k=0}^n {1 \over k!} \left[\begin{matrix} k \\ m \end{matrix}\right]}$\\
${\displaystyle \left[\begin{matrix} n \\ m \end{matrix}\right] = \sum_k \left[\begin{matrix} n+1 \\ k+1 \end{matrix}\right] {k \choose m}(-1)^{m-k}}$~~~~
${\displaystyle \left[\begin{matrix} m+n+1 \\ m \end{matrix}\right] = \sum_{k=0}^m k(n+k)\left[\begin{matrix} n+k \\ k \end{matrix}\right]}$\\
${\displaystyle \left[\begin{matrix} n \\ {\ell+m} \end{matrix}\right] {\ell+m \choose \ell} = \sum_k \left[\begin{matrix} k \\ \ell \end{matrix}\right] \left[\begin{matrix} n-k \\ m \end{matrix}\right] {n \choose k}}$\\


\paragraph{$\blacksquare$ 第二类斯特林数}
\noindent \\
$\left\{\begin{matrix} n \\ k \end{matrix}\right\}$表第二类斯特林数,表基数为$n$的集合的$k$份划分方法数,$$\left\{\begin{matrix} n \\ 1 \end{matrix}\right\}=\left\{\begin{matrix} n \\ n \end{matrix}\right\}=1, \left\{\begin{matrix} n \\ k \end{matrix}\right\}=\left\{\begin{matrix} n-1 \\ k-1 \end{matrix}\right\}+k\left\{\begin{matrix} n-1 \\ k \end{matrix}\right\}$$\\
\begin{tabular}{|c|c|c|c|c|c|c|c|c|}
\hline S(i,j)&1&2&3&4&5&6&7&8\\
\hline 1&1&&&&&&& \\
\hline 2&1&1&&&&&&\\
\hline 3&1&3&1&&&&&\\
\hline 4&1&7&6&1&&&&\\
\hline 5&1&15&25&10&1&&&\\
\hline 6&1&31&90&65&15&1&&\\
\hline 7&1&63&301&350&140&21&1&\\
\hline 8&1&127&966&1701&1050&266&28&1\\
\hline
\end{tabular}\\
$\left\{\begin{matrix} n \\ 2 \end{matrix}\right\}=2^{n-1}-1$~~~~~~~~~~
$\left\{\begin{matrix} n \\ n-1 \end{matrix}\right\}=\binom{n}{2}$~~~~~~~~~~
$\left[\begin{matrix} n \\ k \end{matrix}\right] \geq \left\{\begin{matrix} n \\ k \end{matrix}\right\}$\\
${\displaystyle \left\{\begin{matrix} n+1 \\ m+1 \end{matrix}\right\} = \sum_k {n \choose k} \left\{\begin{matrix} k \\ m \end{matrix}\right\} = \sum_{k=0}^n \left\{\begin{matrix} k \\ m \end{matrix}\right\} (m+1)^{n-k}}$\\
${\displaystyle \left\{\begin{matrix} n \\ m \end{matrix}\right\} = \sum_k {n \choose k} \left\{\begin{matrix} k+1 \\ m+1 \end{matrix}\right\} (-1)^{n-k}}$~~~~~~
${\displaystyle \left\{\begin{matrix} m+n+1 \\ m \end{matrix}\right\} = \sum_{k=0}^m k  \left\{\begin{matrix} n+k \\ k \end{matrix}\right\}}$\\
${\displaystyle {n \choose m} = \sum_k \left\{\begin{matrix} n+1 \\ k+1 \end{matrix}\right\} \left[\begin{matrix} k \\ m \end{matrix}\right] (-1)^{m-k}}$\\
${\displaystyle (n-m)!{n \choose m} = \sum_k \left[\begin{matrix} n+1 \\ k+1 \end{matrix}\right] \left\{\begin{matrix} k \\ m \end{matrix}\right\} (-1)^{m-k}, \quad \forall n \geq m}$\\
${\displaystyle \left\{\begin{matrix} n \\ n-m \end{matrix}\right\} = \sum_k {m-n \choose m+k} {m+n \choose n+k}\left[\begin{matrix} m+k \\ k \end{matrix}\right]}$\\
${\displaystyle\left[\begin{matrix} n \\ n-m \end{matrix}\right] = \sum_k {m-n \choose m+k} {m+n \choose n+k} \left\{\begin{matrix} m+k \\ k \end{matrix}\right\}}$\\
${\displaystyle \left\{\begin{matrix} n \\ \ell+m \end{matrix}\right\} {\ell+m \choose \ell} = \sum_k \left\{\begin{matrix} k \\ \ell \end{matrix}\right\} \left\{\begin{matrix} n-k \\ m \end{matrix}\right\} {n \choose k}}$\\

\paragraph{$\blacksquare$ 贝尔数}
\noindent \\
$B_n$表贝尔数,表基数为$n$的集合的划分方法数,$${\displaystyle B_0=1, B_{n+1}=\sum_{k=0}^{n}{\binom{n}{k} B_k}}$$\\
\begin{tabular}{|c|c|c|c|c|c|c|c|c|c|c|c|c|c|}
\hline $n$&0&1&2&3&4&5&6&7&8&9&10&11\\
\hline $B_n$&1&1&2&5&15&52&203&877&4140&21147&115975&678570\\
\hline
\end{tabular}\\
${\displaystyle B_n=\sum _{k=1}^{n} \left\{\begin{matrix} n \\ k \end{matrix}\right\}}$~~~~~~
${\displaystyle B_n=\frac{1}{e}\sum_{k=0}^\infty \frac{k^n}{k!}}$~~~~~~~~
${\displaystyle \sum_{n=0}^\infty \frac{B_n}{n!} x^n = e^{e^x-1}}$\\
$p$是质数$\Rightarrow B_{n+p}\equiv B_n+B_{n+1}~(\operatorname{mod}~p)$\\
\paragraph{$\blacksquare$ 卡特兰数}
\noindent \\
$C_n$表卡特兰数,$$C_n=\frac{1}{n+1}\binom{2n}{n} \quad n\geq 0$$\\
\begin{tabular}{|c|c|c|c|c|c|c|c|c|c|c|c|c|}
\hline $n$&0&1&2&3&4&5&6&7&8&9&10&11\\
\hline $C_n$&1&1&2&5&14&42&132&429&1430&4862&16796&58786\\
\hline
\end{tabular}\\
$C_n=\binom{2n}{n}-\binom{2n}{n+1} \quad \forall n\geq 1$~~~~~~~~
${\displaystyle C_{n+1}=\sum _{k=0}^{n} C_k C_{n-k} \quad \forall n\geq 0}$\\
$C_{n+1}=\frac{4n+2}{n+2} C_n$~~~~~~~~
$C_n$为奇数$\Leftrightarrow n=2^k-1, k\in \mathbb{Q}$\\
大小为$n$的不同构二叉树数目为$C_n$;$n\times n$格点不越过对角线的单调路径(比如仅向右或上)数目为$C_n$;$n+2$边凸多边形分成三角形的方法数为$C_n$;高度为$n$的阶梯形分成n个长方形的方法数为$C_n$;待进栈的$n$个元素的出栈序列种数为$C_n$
\paragraph{$\blacksquare$ 伯努利数}
\noindent \\
$b_n$表$n$次伯努利数,$${\displaystyle b_0=1, \sum _{k=0}^{m} \binom{m+1}{k} b_k=0}$$\\
\begin{tabular}{|c|c|c|c|c|c|c|c|c|c|c|c|c|c|}
\hline $n$&0&1&2&3&4&5&6&7&8&9&10&11&12\\
\hline $b_n$&1&$-\frac{1}{2}$&$\frac{1}{6}$&0&$-\frac{1}{30}$&0&$\frac{1}{42}$&0&$-\frac{1}{30}$&0&$\frac{5}{66}$&0&$-\frac{691}{2730}$\\
\hline
\end{tabular}\\
\paragraph{$\blacksquare$ 斐波那契数列}
\noindent \\
$F_n$表斐波那契数列,$F_0=0, F_1=F_2=1, F_n=F_{n-1}+F_{n-2}$\\

\subsubsection{泰勒级数}
\def\Subset#1#2{ \bigg\{{#1 \atop #2} \bigg\}}
\def\subset#1#2{ \big\{{#1 \atop #2} \big\}}
\def\Cycle#1#2{ \bigg[{#1 \atop #2} \bigg]}
\def\cycle#1#2{ \big[{#1 \atop #2} \big]}

$$f(x)=\sum _{n=0}^{\infty }{\frac {f^{(n)}(a)}{n!}}(x-a)^{n}$$
${\displaystyle {1 \over 1 - x}=\sum_{i=0}^\infty x^i}$~~~~
${\displaystyle {1 \over 1 - c x}=\sum_{i=0}^\infty c^i x^i}$~~~~
${\displaystyle {1 \over 1 - x^n}=\sum_{i=0}^\infty x^{ni}}$~~~~
${\displaystyle {x \over (1 - x)^2}=\sum_{i=0}^\infty i x^i}$\\
${\displaystyle\sum_{k=0}^n {n \brace k} {k! z^k \over (1-z)^{k+1}}=\sum_{i=0}^\infty i^n x^i}$~~~~~~~~~~~~~~~~~~~~~
${\displaystyle e^x=\sum_{i=0}^\infty {x^i \over i!}}$\\
${\displaystyle \ln (1 + x)=\sum_{i=1}^\infty (-1)^{i+1} {x^i \over i}}$~~~~~~~~~~~~~~~~~~~~~~~~~~~
${\displaystyle \ln {1 \over 1 - x}=\sum_{i=1}^\infty {x^i \over i}}$\\
${\displaystyle \sin x=\sum_{i=0}^\infty (-1)^i {x^{2i+1} \over (2i+1)!}}$~~~~~~~~~~~~~~~~~~~~~~~~~~~~
${\displaystyle \cos x=\sum_{i=0}^\infty (-1)^i {x^{2i} \over (2i)!} }$\\
${\displaystyle \tan^{-1} x=\sum_{i=0}^\infty (-1)^i {x^{2i+1} \over (2i+1)} }$~~~~~~~~~~~~~~~~~~~~~~~~~
${\displaystyle (1+x)^n=\sum_{i=0}^\infty {n \choose i} x^i }$\\
${\displaystyle {1 \over (1-x)^{n+1}}=\sum_{i=0}^\infty {i+ n \choose i} x^i }$~~~~~~~~~~~~~~~~~~~~~~~~~
${\displaystyle {x \over e^x - 1}=\sum_{i=0}^\infty {b_i x^i \over i!} }$\\
${\displaystyle {1 \over 2x}(1 - \sqrt{1-4x})=\sum_{i=0}^\infty {1 \over i+1}{2i \choose i}x^i }$~~~~~~~~~~~~~~
${\displaystyle {1 \over \sqrt{1-4x}}=\sum_{i=0}^\infty {2i \choose i}x^i }$\\
${\displaystyle {1 \over \sqrt{1-4x}}\left({1 - \sqrt{1-4x} \over 2x}\right)^n=\sum_{i=0}^\infty {2i+n \choose i}x^i }$~~~
${\displaystyle {1 \over 1-x}\ln{1 \over 1- x}=\sum_{i=1}^\infty H_i x^i}$\\
${\displaystyle {1 \over 2}\left(\ln{1 \over 1- x}\right)^2=\sum_{i=2}^\infty {H_{i-1} x^i \over i}}$~~~~~~~~~~~~~~~~~~~~~~~~
${\displaystyle {x \over 1 - x - x^2}=\sum_{i=0}^\infty F_i x^i}$\\
${\displaystyle {F_n x \over 1 - (F_{n-1} + F_{n+1})x - (-1)^n x^2}=\sum_{i=0}^\infty F_{ni} x^i.}$\\
${\displaystyle {1 \over (1-x)^{n+1}}\ln{1 \over 1- x} = \sum_{i=0}^\infty (H_{n+i} - H_n) {n+i \choose i} x^i, \left({1 \over x}\right)^{\overline{-n}} = \sum_{i=0}^\infty \Subset i n x^i}$\\
${\displaystyle x^{\overline{n}} = \sum_{i=0}^\infty \Cycle n i x^i, (e^x - 1)^n = \sum_{i=0}^\infty \Subset i n {n! x^i \over i!}}$\\
${\displaystyle \left(\ln {1 \over 1 -x}\right)^n = \sum_{i=0}^\infty \Cycle i n {n! x^i \over i!}, x \cot x = \sum_{i=0}^\infty {(-4)^i b_{2i} x^{2i} \over (2i)!}}$\\
${\displaystyle \tan x = \sum_{i=1}^\infty (-1)^{i-1}{2^{2i} (2^{2i} - 1) b_{2i} x^{2i-1} \over (2i)!}, \zeta(x) = \sum_{i=1}^\infty {1 \over i^x}}$\\
${\displaystyle {1 \over \zeta(x)} = \sum_{i=1}^\infty {\mu(i) \over i^x}, {\zeta(x-1) \over \zeta(x)} = \sum_{i=1}^\infty {\phi(i) \over i^x}}$\\

\subsubsection{导数}
\paragraph{$\blacksquare$ 几个导数}
\noindent \\
$(\tan x)'=\sec ^2 x$~~~
$(\arctan x)'=\frac{1}{1+x^2}$~~~
$(\arcsin x)'=\frac{1}{\sqrt{1-x^2}}$~~~
$(\arccos x)'=-\frac{1}{\sqrt{1-x^2}}$\\
$(\sinh x)'=\cosh x=\frac{e^x+e^{-x}}{2}$~~~~
$(\cosh x)'=\sinh x=\frac{e^x-e^{-x}}{2}$
\paragraph{$\blacksquare$ 高阶导数}
\noindent \\
(莱布尼茨公式)
$$(uv)^{(n)}=\sum _{{k=0}}^{n}{\binom{n}{k} u^{(n-k)} v^{(k)}}$$
${\displaystyle (x^a)^{(n)}=x^{a-n} \prod _{{k=0}}^{{n-1}} (a-k)}$~~~~~~~~
$(\frac{1}{x})^{(n)}=(-1)^n\frac{n!}{x^{n+1}}$\\
$(a^x)^{(n)}=a^x \ln ^n a~(a>0)$~~~~~~~~
$(\ln x)^{(n)}=(-1)^{n-1} \frac{(n-1)!}{x^n}$\\
$(\sin (kx+b))^{(n)}=k^n \sin (kx+b+\frac{n\pi}{2})$~~~~
$(\cos (kx+b))^{(n)}=k^n \cos (kx+b+\frac{n\pi}{2})$

\subsubsection{积分表}
\input{Formula/integrals.tex}

\subsubsection{其它}
\paragraph{$\blacksquare$ 克拉夫特不等式}
若二叉树有$n$个叶子,深度分别为$d_1, d_2, ... ,d_n$,则${\displaystyle \sum _{i=1}^{n} 2^{-d_i} \leq 1}$,当且仅当叶子都有兄弟时取等



\subsection{几何公式}
\input{Formula/geometry.tex}

\subsection{经典博弈}
\paragraph{$\blacksquare$ Nim博弈\\}
\noindent \\
问题:$n$堆石子,每次取一堆中$x$个($x>0$),取完则胜。\\
奇异态(后手胜):$a_1~xor~a_2~xor~...~xor~a_n=0$
\paragraph{$\blacksquare$ Bash博弈\\}
\noindent \\
问题:$n$个石子,每次$x$个($0<x\leq m$),取完则胜。\\
奇异态(后手胜):$n\equiv 0~(\operatorname{mod}~(m+1))$
\paragraph{$\blacksquare$ Wythoff博弈\\}
\noindent \\
问题:2堆石子分别$x, y$个($x>y$),每次取一堆中$x$个($x>0$),或两堆中分别$x$个($x>0$),取完则胜。\\
奇异态(后手胜):$\left \lfloor \frac{\sqrt{5}+1}{2} (x-y) \right \rfloor =y$
\paragraph{$\blacksquare$ Fibonacci博弈\\}
\noindent \\
问题:$n$个石子,先手第一次取$x$个($0<x<n$),之后每次取$x$个($0<x\leq$上一次取数的两倍),取完则胜。\\
奇异态(\textbf{先}手胜):$n$不是斐波那契数


\subsection{部分质数}
\noindent
$100003$, $200003$, $300007$, $400009$, $500009$, $600011$, $700001$, $800011$, $900001$, \\
$1000003$, $2000003$, $3000017$, $4100011$, $5000011$, $8000009$, $9000011$, \\
$10000019$, $20000003$, $50000017$, $50100007$, \\
$100000007$, $100200011$, $200100007$, $250000019$
%==================================================
\section{语法}
精选部分函数,无特别说明则为98标准
\subsection{C}
\subsubsection{<cstdio>}
\lstinputlisting{Syntax/cstdio.cpp}
\subsubsection{<cctype>}
\lstinputlisting{Syntax/cctype.cpp}
\subsubsection{<cstring>}
\lstinputlisting{Syntax/cstring.cpp}
\subsubsection{<cstdlib>}
\lstinputlisting{Syntax/cstdlib.cpp}
\subsubsection{无头文件}
\lstinputlisting{Syntax/c.cpp}

%under construction!
\subsection{C++}
\subsubsection{<iostream>/<ios>}
\lstinputlisting{Syntax/iostream_ios.cpp}
\subsubsection{Containers}
\lstinputlisting{Syntax/containers.cpp}
\subsubsection{<string>}
\lstinputlisting{Syntax/string.cpp}
%\subsubsection{<algorithm>}
%\lstinputlisting{Syntax/algorithm.cpp}
%\subsubsection{<bitset>}
%\lstinputlisting{Syntax/bitset.cpp}
%\subsubsection{<complex>}
%\lstinputlisting{Syntax/complex.cpp}
\subsubsection{pb_ds}
\lstinputlisting{Syntax/pb_ds.cpp}
\subsubsection{无头文件}
\lstinputlisting{Syntax/c++.cpp}
%==================================================
\section{经典错误}
=和==混淆;scanf没加\&;爆数组/数据范围
\end{document}
