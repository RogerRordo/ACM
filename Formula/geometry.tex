\subsubsection{平面几何}
\paragraph{$\blacksquare$ 三角形的长度}
\noindent \\
中线$m_a=\sqrt{\frac{1}{2} b^2+\frac{1}{2} c^2-\frac{1}{4} a^2}$\\
高线长$h_a=\frac{2\sqrt{s(s-a)(s-b)(s-c)}}{a}$\\
角平分线$t_a=\frac{1}{b+c} \sqrt{(b+c+a)(b+c-a)bc}$\\
外接圆半径$R=\frac{abc}{\sqrt{(a+b+c)(b+c-a)(a+c-b)(a+b-c)}}$\\
内切圆半径$r=\frac{\sqrt{(a+b+c)(b+c-a)(a+c-b)(a+b-c)}}{2(a+b+c)}$
\paragraph{$\blacksquare$ 三角形的面积}
\noindent \\
$S=\frac{1}{2} ab\sin C=\frac{a^2\sin B\sin C}{2\sin (B+C)}=\sqrt{p(p-a)(p-b)(p-c)}=\frac{1}{2}\begin{Vmatrix} a_x & a_y & 1\\ b_x & b_y & 1\\ c_x & c_y & 1 \end{Vmatrix}$,其中$p=\frac{a+b+c}{2}$
\paragraph{$\blacksquare$ 三角形奔驰定理}
\noindent \\
$P$为$\triangle ABC$中一点,且$S_{\triangle PBC} \cdot \overrightarrow{PA}+S_{\triangle PAC} \cdot \overrightarrow{PB}+S_{\triangle PAB} \cdot \overrightarrow{PC}=\vec 0$
\paragraph{$\blacksquare$ 托勒密定理}
\noindent \\
狭义:凸四边形四点共圆当且仅当其两对对边乘积的和等于两条对角线的乘积\\
广义:四边形$ABCD$两条对角线长分别为$m, n$,则$m^2 n^2=a^2 c^2+b^2 d^2-2abcd \cos(A+C)$
\paragraph{$\blacksquare$ 椭圆面积}
$S=\pi ab$
\paragraph{$\blacksquare$ 弧微分}
$\mathrm{d}s=\sqrt{[x'(t)]^2+[y'(t)]^2} \mathrm{d} t=\sqrt{1+[f'(x)]^2} \mathrm{d} x=\sqrt{r^2(\theta)+[r'(\theta)]^2} \mathrm{d} \theta$
\paragraph{$\blacksquare$ 费马点}
三角形费马点是指与三顶点距离之和最小的点。当有一个内角不小于120°时,费马点为此角对应顶点;当三角形的内角都小于120°时,据三角形各边向外做正三角形,连接新产生的三点与各自在原三角形中所对顶点,则三线交于费马点。

\subsubsection{立体几何}
\paragraph{$\blacksquare$ 凸多面体欧拉公式}
对任意凸多面体,点、边、面数分别为$V, E, F$,则$V-E+F=2$
\paragraph{$\blacksquare$ 台体体积}
$V=\frac{1}{3}h(S_1+\sqrt {S_1S_2}+S_2)$
\paragraph{$\blacksquare$ 椭球体积}
$V=\frac{4}{3} \pi abc$(都是半轴)
\paragraph{$\blacksquare$ 四面体体积}
\noindent \\
$V=\frac{1}{6} \begin{Vmatrix} p_x & p_y & p_z\\ q_x & q_y & q_z\\ r_x & r_y & r_z \end{Vmatrix}$,其中$\vec p=\overrightarrow{OA}, \vec q=\overrightarrow{OB}, \vec r=\overrightarrow{OC}$;\\
$(12V)^2=a^2d^2(b^2+c^2+e^2+f^2-a^2-d^2)+b^2e^2(c^2+a^2+f^2+d^2-b^2-e^2)+c^2f^2(a^2+b^2+d^2+e^2-c^2-f^2)-a^2b^2c^2-a^2e^2f^2-d^2b^2f^2-d^2e^2c^2$,其中$a=AB,b=BC,c=CA,d=OC,e=OA,f=OB$
\paragraph{$\blacksquare$ 旋转体(一、二象限,绕x轴)}
\noindent \\
体积$V=\pi \int_{a}^{b} f^2(x) \mathrm{d} x$\\
侧面积$F=2\pi \int f(x) \mathrm{d} s=2\pi \int_{a}^{b} \sqrt{1+[f'(x)]^2} \mathrm{d} x$\\
(空心)质心
\begin{align*}
	X&=\frac{1}{M} \int_{\alpha}^{\beta} x(t) \rho (t) \sqrt{[x'(t)]^2+[y'(t)]^2} \mathrm{d} t\\
	Y&=\frac{1}{M} \int_{\alpha}^{\beta} y(t) \rho (t) \sqrt{[x'(t)]^2+[y'(t)]^2} \mathrm{d} t\\
\end{align*}
(空心)转动惯量
\begin{align*}
	J_x&=\int_{\alpha}^{\beta} y^2(t) \rho (t) \sqrt{[x'(t)]^2+[y'(t)]^2} \mathrm{d} t\\
	J_y&=\int_{\alpha}^{\beta} x^2(t) \rho (t) \sqrt{[x'(t)]^2+[y'(t)]^2} \mathrm{d} t\\
\end{align*}
古鲁丁定理:平面上一条质量分布均匀曲线绕一条不通过它的直线轴旋转一周,所得到的旋转体之侧面积等于它的质心绕同一轴旋转所得圆的周长乘以曲线的弧长。
